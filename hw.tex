\documentclass[12pt]{article}

\usepackage{cmap}
\usepackage[T2A]{fontenc}
\usepackage[utf8]{inputenc}
\usepackage[russian]{babel}
\usepackage{graphicx}
\usepackage{amsthm,amsmath,amssymb}
\usepackage[russian,colorlinks=true,urlcolor=red,linkcolor=blue]{hyperref}
\usepackage{enumerate}
\usepackage{datetime}
\usepackage{minted}
\usepackage{fancyhdr}
\usepackage{lastpage}
\usepackage{color}
\usepackage{verbatim}


\def\WHOAMI{Воронин Илья}
\def\DATE{26 марта}
\def\CURNO{\NO\t{1}}

\parskip=0em
\parindent=0em

\sloppy
\voffset=-20mm
\textheight=235mm
\hoffset=-25mm
\textwidth=180mm
\headsep=12pt
\footskip=20pt

\setcounter{page}{0}
\pagestyle{empty}

% Shortcuts for writing solutions
\newcommand{\solution}[1]{
      \\ {\bf Решение.} \\
      \input{solutions/#1}
      \vspace*{0.5em}
}

% Основные математические символы
\DeclareSymbolFont{extraup}{U}{zavm}{m}{n}
\DeclareMathSymbol{\heart}{\mathalpha}{extraup}{86}
\newcommand{\N}{\mathbb{N}}   % Natural numbers
\newcommand{\R}{\mathbb{R}}   % Real numbers
\newcommand{\Z}{\mathbb{Z}}   % Integer numbers
\def\INF{\t{+}\infty}         % +inf
\def\EPS{\varepsilon}         %
\def\EMPTY{\varnothing}       %
\def\PHI{\varphi}             %
\def\SO{\Rightarrow}          % =>
\def\EQ{\Leftrightarrow}      % <=>
\def\t{\texttt}               % mono font
\def\c#1{{\rm\sc{#1}}}        % font for classes NP, SAT, etc
\def\O{\mathcal{O}}           %
\def\NO{\t{\#}}               % #
\renewcommand{\le}{\leqslant} % <=, beauty
\renewcommand{\ge}{\geqslant} % >=, beauty
\def\XOR{\text{ {\raisebox{-2pt}{\ensuremath{\Hat{}}}} }}
\newcommand{\q}[1]{\langle #1 \rangle}               % <x>
\newcommand\URL[1]{{\footnotesize{\url{#1}}}}        %
\newcommand{\sfrac}[2]{{\scriptstyle\frac{#1}{#2}}}  % Очень маленькая дробь
\newcommand{\mfrac}[2]{{\textstyle\frac{#1}{#2}}}    % Небольшая дробь
\newcommand{\score}[1]{{\bf\color{red}{(#1)}}}

% Отступы
\def\makeparindent{\hspace*{\parindent}}
\def\up{\vspace*{-0.3em}}
\def\down{\vspace*{0.3em}}
\def\LINE{\vspace*{-1em}\noindent \underline{\hbox to 1\textwidth{{ } \hfil{ } \hfil{ } }}}
%\def\up{\vspace*{-\baselineskip}}

\lhead{Формальные языки, весна 2019/20}
\chead{}
\rhead{Практика \CURNO.}
\renewcommand{\headrulewidth}{0.4pt}

\lfoot{}
\cfoot{\thepage\t{/}\pageref*{LastPage}}
\rfoot{}
\renewcommand{\footrulewidth}{0.4pt}

\newenvironment{MyList}[1][4pt]{
  \begin{enumerate}[1.]
  \setlength{\parskip}{0pt}
  \setlength{\itemsep}{#1}
}{       
  \end{enumerate}
}
\newenvironment{InnerMyList}[1][0pt]{
  \vspace*{-0.5em}
  \begin{enumerate}[a)]
  \setlength{\parskip}{#1}
  \setlength{\itemsep}{0pt}
}{
  \end{enumerate}
}

\newcommand{\Section}[1]{
  \refstepcounter{section}
  \addcontentsline{toc}{section}{\arabic{section}. #1} 
  %{\LARGE \bf \arabic{section}. #1} 
  {\LARGE \bf #1} 
  \vspace*{1em}
  \makeparindent\unskip
}
\newcommand{\Subsection}[1]{
  \refstepcounter{subsection}
  \addcontentsline{toc}{subsection}{\arabic{section}.\arabic{subsection}. #1} 
  {\Large \bf \arabic{section}.\arabic{subsection}. #1} 
  \vspace*{0.5em}
  \makeparindent\unskip
}

% Код с правильными отступами
\newenvironment{code}{
  \VerbatimEnvironment

  \vspace*{-0.5em}
  \begin{minted}{c}}{
  \end{minted}
  \vspace*{-0.5em}

}

% Формулы с правильными отступами
\newenvironment{smallformula}{
 
  \vspace*{-0.8em}
}{
  \vspace*{-1.2em}
  
}
\newenvironment{formula}{
 
  \vspace*{-0.4em}
}{
  \vspace*{-0.6em}
  
}

\definecolor{dkgreen}{rgb}{0,0.6,0}
\definecolor{brown}{rgb}{0.5,0.5,0}
\newcommand{\red}[1]{{\color{red}{#1}}}
\newcommand{\dkgreen}[1]{{\color{dkgreen}{#1}}}

\begin{document}
\renewcommand{\dateseparator}{--}
\begin{center}
  {\Large\bf 
   Второй курс, Весенний семестр 2019/20\\
   Дз по формальным языкам \CURNO\\
   \vspace*{0.5em} 
   }
  \vspace{0.5em}
  {\WHOAMI}
\end{center}

\vspace{-1em}
\LINE
\vspace{1em}


\pagestyle{fancy}


\begin{enumerate}
	    \item [\bf \textnumero 2]
	    1)Сначала удалим длинные правила:\\
	      1. S -> RS | R\\
	      2. R -> ab | cd | $\epsilon$\\
	      3. R -> aA\\
	      4. A -> Sb\\
	      5. R -> cB\\
	      6. B -> Rd\\
	    2)Теперь удалим эпсилон правила:\\
	       Так как из S можем получит $\epsilon$, то создадим новый стартовый нетерминал. $\epsilon$ достижимо из S и из R:\\
	        1. S -> C | $\epsilon$\\
	        2. C -> RC | R\\
			3. R -> ab | cd
			4. R -> aA\\
			5. A -> Cb\\
			6. R -> cB\\
			7. B -> Rd\\
			Новые правила:\\
			8. B -> d\\
			9. A -> b\\
	     3) Удалим цепные правила. В данном случае это С->R, S->C.  Цепные пары: (S, C), (S, R), (C, R):\\
	     	 1. S -> $\epsilon$\\
		     2. C -> RC\\
		     3. R -> ab | cd
		     4. R -> aA\\
		     5. A -> Cb | b\\
		     6. R -> cB\\
		     7. B -> Rd\\
		     8. B -> d\\
		     Новые правила:\\
		     9)S -> RC | ab | cd | aA | cB\\
		     10)C -> ab | cd | aA | cB\\
		  4) Объединим правила для одинаковых нетерминалов:
		     1. S -> RC | ab | cd | aA | cB | $\epsilon$\\
		     2. C -> RC | ab | cd | aA | cB \\
		     3. R -> ab | cd | aA | cB\\
		     4. A -> Cb | b\\
		     5. B -> Rd | d\\
		  5) Уберем двойные правила в которых есть терминал(a заменим на A'', b на B'', c на C'', d на D''):\\
	    		1. S -> RC | A''B'' | C''D'' | A''A | C''B | $\epsilon$\\
			    2. C -> RC | A''B'' | C''D'' | A''A | C''B \\
			    3. R -> A''B'' | C''D'' | A''A | C''B\\
			    4. A -> CB'' | b\\
			    5. B -> RD'' | d\\
			    6. A'' -> a\\
			    7. B'' -> b\\
			    8) C'' -> c\\
			    9) D'' -> d\\
	    \item [\bf \textnumero 3]
	    Является. Вот КС грамматика:\\
	      1. S -> E | O\\
	      2. E -> AA | BB | AABB\\
	      3. O -> AB\\
	      4. B -> BBB\\
	      5. A -> AAA\\
	      6. A -> a\\
	      7. B -> b\\
	      Докажем что можем получить все слова. Пусть есть слово $a^nb^m$. Если n и m четные и больше 0, то цепочка такая S -> E -> AABB -> (потом добавляем по два к A или B до нужного количества правилами 4,5). Если n = 0, то m точно четное и цепочка S -> BB -> (так же как в первом случае). Аналогично когда m = 0. Если оба нечетные то S -> AB -> (как в первом случае).
\end{enumerate}


\end{document}
